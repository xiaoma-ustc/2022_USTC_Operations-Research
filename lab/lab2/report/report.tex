\documentclass[12pt, a4paper, oneside]{ctexart}
\usepackage{amsmath, amsthm, amssymb, graphicx}
\usepackage{hyperref}
\usepackage{listings}
\usepackage{xcolor}
\usepackage{color}
\usepackage{enumerate}
\usepackage{epstopdf}
\usepackage{float}
\usepackage{booktabs}
\usepackage[ruled,vlined]{algorithm2e}
\hypersetup{
    colorlinks=true,
    linkcolor=blue,
    filecolor=blue,      
    urlcolor=blue,
    citecolor=cyan,
}
\definecolor{dkgreen}{rgb}{0,0.6,0}
\definecolor{gray}{rgb}{0.5,0.5,0.5}
\definecolor{mauve}{rgb}{0.58,0,0.82}

\lstset{ %
    language=Python,                % the language of the code
    basicstyle=\footnotesize,           % the size of the fonts that are used for the code
    numbers=left,                   % where to put the line-numbers
    %numberstyle=\tiny\color{gray},  % the style that is used for the line-numbers
    %stepnumber=2,                   % the step between two line-numbers. If it's 1, each line 
                            % will be numbered
    %numbersep=5pt,                  % how far the line-numbers are from the code
    %backgroundcolor=\color{blue},      % choose the background color. You must add \usepackage{color}
    showspaces=false,               % show spaces adding particular underscores
    %showstringspaces=false,         % underline spaces within strings
    showtabs=false,                 % show tabs within strings adding particular underscores
    frame=single,                   % adds a frame around the code
    rulecolor=\color{black},        % if not set, the frame-color may be changed on line-breaks within not-black text (e.g. commens (green here))
    tabsize=2,                      % sets default tabsize to 2 spaces
    captionpos=b,                   % sets the caption-position to bottom
    breaklines=true,                % sets automatic line breaking
    breakatwhitespace=false,        % sets if automatic breaks should only happen at whitespace
    title=\lstname,                   % show the filename of files included with \lstinputlisting;
                            % also try caption instead of title
    keywordstyle=\color{blue},          % keyword style
    commentstyle=\color{dkgreen},       % comment style
    stringstyle=\color{mauve},         % string literal style
    escapeinside={\%*}{*)},            % if you want to add LaTeX within your code
    morekeywords={*,...}               % if you want to add more keywords to the set
}
\title{基于Wolfe-Powell准则的非精确一维步长搜索算法}
\author{Xiaoma}
\date{\today}
\begin{document}
\maketitle
\section*{问题描述}
给定目标函数$f(x)$:
\begin{itemize}
    \item 二维Rosenbrock函数:
    $$f(x) = 100(x_{1}^{2}-x_{2})^{2}+(x_{1} - 1)^{2}$$
    

\end{itemize}
求解无约束优化问题:
$$\min_{x} f(x)$$

\section*{实验原理}
\subsection*{拟牛顿法}
\subsubsection*{牛顿法}
牛顿法的基本思想是利用目标函数的二次Taylor展开,并将其极小化。

设$f(x)$是可微实函数,$x^{(k)} \in \mathbb{R}^{n}$,Hesse矩阵$\nabla^{2}f(x^{(k)})$正定,在$x^{(k)}$附近用二次
Taylor展开近似$f$
$$q^{(k)}(s) = f(x^{(k)}) + \nabla f(x^{(k)})^{T}s + \frac{1}{2} s^{T}\nabla^{2}f(x^{(k)})s$$
其中$s=x-x^{(k)}$,$q^{(k)}(s)$为$f(x)$的二次近似,将上式右边极小化得
$$x^{(k + 1)} = x^{(k)} - [\nabla^{2}f(x^{(k)})]^{-1}\nabla f(x^{(k)})$$
在该公式中,步长因子$a_{k}=1$,令$G_{k} = \nabla^{2}f(x^{(k)}),g^{(k)} = \nabla f(x^{k})$,则原式可写为
$$x^{(k + 1)} = x^{(k)} - G_{k}^{-1}g^{(k)}$$
显然,牛顿法也可看成在椭球范数$\| \cdot \Vert_{G_{k}} $下的最速下降法,对于$f(x^{k} + s) \thickapprox  f(x^{(k)}) + g^{(k)\mathcal{T}}s$,$s^{(k)}$
是极小化问题
$$\min \frac{g^{(k)\mathcal{T}}}{\| s\Vert}$$
的解,当采用$l_{2}$范数时
$$s^{(k)} = -g^{(k)}$$
所得方法是最速下降法,当采用椭球范数$\| \cdot \Vert_{G_{k}} $时
$$s^{(k)} = -G_{k}^{-1}g^{(k)}$$
所得方法是牛顿法。

经典牛顿迭代法的运算步骤为
\begin{algorithm*}
    \caption{\text{牛顿法}}
    \label{alg:algorithm}
    \KwIn{\text{初始点$x_{0}$,阈值误差$\varepsilon $}}
    \KwOut{\text{$x^{(k)},f(x^{(k)})$}}
    \BlankLine
    (1) 初始化k=0

    (2) 计算 $g^{(k)} = \nabla f(x^{(k)})$,如果$\| g^{(k)}\Vert <\varepsilon $,则停止迭代

    (3) 解线性方程组$s^{(k)} = -G_{k}^{-1}g^{(k)}$

    (4) 更新$x^{(k+1)} = x^{(k)} + s^{(k)},k = k + 1$,返回步骤(2)
\end{algorithm*}
\subsubsection*{拟牛顿法}

牛顿法虽然收敛速度较快,但计算Hense矩阵的成本过大,并且若矩阵不是正定的,则牛顿法失效。

对$\nabla f(x)$在$x^{(k)}$出Taylor展开得到如下近似
$$
\nabla f(x) = g^{(k)} + H_{k}(x - x^{(k)})
$$
令$x = x^{(k+1)}$即可得到$g^{(k+1)} - g^{(k)}=H_{k}(x^{(k+1)}-x^{(k)})$\\
记$y^{(k)}= g^{(k+1)} - g^{(k)}, \delta _{k} = x^{(k+1)} - x^{(k)}$
$$
y^{(k)} = H_{k} \delta_{k}
$$
称为拟牛顿条件

\subsubsection*{DFP法}
设对称秩二矫正为
$$
H_{k+1} = H_{k} + auu^{T} + bvv^{T}
$$
令拟牛顿条件满足,则
$$
H_{k}y^{(k)} + auu^{T}y^{(k)} + bvv^{T}y^{(k)} = s^{(k)}
$$
这里u和v并不唯一确定,但u和v的明显选择是
$$
u = s^{(k)} \quad v = H_{k}y^{(k)}
$$
确定出
$$
a = 1 / y^{(k)}u^{T} = 1/s^{(k)T}y^{(k)}\\
b = -1v^{T}/y^{(k)} = - 1/y^{(k)T}H_{k}y^{(k)}
$$
因此
$$
H_{k+1} = H_{k} + \frac{s^{(k)}s^{(k)T}}{s^{(k)T}y^{(k)}} - \frac{H_{k}y^{(k)}y^{(k)T}H_{k}}{y^{(k)T}H_{k}y^{(k)}}
$$
这个公式称为DFP公式。
DFP法的运算步骤为
\begin{algorithm*}
    \caption{\text{DFP}}
    \label{alg:algorithm}
    \KwIn{\text{初始点$x_{0}$,阈值误差$\varepsilon $}}
    \KwOut{\text{$x^{(k)},f(x^{(k)})$}}
    \BlankLine
    (1) 初始化$H_{0} = I, k = 0$

    (2) 计算搜索方向:$d^{(k)} = -H_{k} \nabla f(x^{(k)})$,如果$\| g^{(k)}\Vert < \varepsilon$,停止迭代

    (3) 一维搜索确定步长$\alpha_{k}$,令$x^{(k+1)} = x^{(k)} + \alpha_{k}d^{(k)}$

    (4) 令 $\mathbf{s}^{(k)}=\mathbf{x}^{(k+1)}-\mathbf{x}^{(k)}, \mathbf{y}^{(k)}=\nabla f\left(\mathbf{x}^{(k+1)}\right)-\nabla f\left(\mathbf{x}^{(k)}\right)$, 当 $\mathbf{s}^{(k)^T} \mathbf{y}^{(k)}>0$ ,作更新 $H_{k+1}=H_k+\frac{\mathbf{s}^{(k)} \mathbf{s}^{(k)^T}}{\mathbf{s}^{(k)^T} \mathbf{y}^{(k)}}-\frac{H_k \mathbf{y}^{(k)} \mathbf{y}^{(k){ }^T} H_k}{\mathbf{y}^{(k)^T} H_k \mathbf{y}^{(k)}}$ 。置 $k=k+1$, 返回步骤 $(2) 。$
\end{algorithm*}

\subsubsection*{BFGS}
类似的,可以得到关于$B_{k}$的对称秩二矫正公式
$$
B_{k+1}^{(B F G S)}=B_k+\frac{\mathbf{y}^{(k)} \mathbf{y}^{(k)^T}}{\mathbf{y}^{(k)^T} \mathbf{s}^{(k)}}-\frac{B_k \mathbf{s}^{(k)} \mathbf{s}^{(k)^T} B_k}{\mathbf{s}^{(k)^T} B_k \mathbf{s}^{(k)}}
$$
$H_{k}$的BFGS校正公式为
$$
\begin{aligned}
H_{k+1}^{(B F G S)}= & H_k+\left(1+\frac{\mathbf{y}^{(k)^T} H_k \mathbf{y}^{(k)}}{\mathbf{s}^{(k)^T} \mathbf{y}^{(k)}}\right) \frac{\mathbf{s}^{(k)} \mathbf{s}^{(k)^T}}{\mathbf{s}^{(k)^T} \mathbf{y}^{(k)}} \\
& -\frac{H_k \mathbf{y}^{(k)} \mathbf{s}^{(k)^T}+\mathbf{s}^{(k)} \mathbf{y}^{(k)^T} H_k}{\mathbf{s}^{(k)^T} \mathbf{y}^{(k)}} .
\end{aligned}
$$
BFGS法的运算步骤为
\begin{algorithm*}
    \caption{\text{BFGS}}
    \label{alg:algorithm}
    \KwIn{\text{初始点$x_{0}$,阈值误差$\varepsilon $}}
    \KwOut{\text{$x^{(k)},f(x^{(k)})$}}
    \BlankLine
    (1) 初始化$H_{0} = I, k = 0$

    (2) 计算搜索方向:$d^{(k)} = -H_{k} \nabla f(x^{(k)})$,如果$\| g^{(k)}\Vert < \varepsilon$,停止迭代

    (3) 一维搜索确定步长$\alpha_{k}$,令$x^{(k+1)} = x^{(k)} + \alpha_{k}d^{(k)}$

    (4) 令 $\mathbf{s}^{(k)}=\mathbf{x}^{(k+1)}-\mathbf{x}^{(k)}, \mathbf{y}^{(k)}=\nabla f\left(\mathbf{x}^{(k+1)}\right)-\nabla f\left(\mathbf{x}^{(k)}\right)$ ,作更新 $H_{k+1}^{(B F G S)}=H_k+\left(1+\frac{\mathbf{y}^{(k)^T} H_k \mathbf{y}^{(k)}}{\mathbf{s}^{(k)^T} \mathbf{y}^{(k)}}\right) \frac{\mathbf{s}^{(k)} \mathbf{s}^{(k)^T}}{\mathbf{s}^{(k)^T} \mathbf{y}^{(k)}}-\frac{H_k \mathbf{y}^{(k)} \mathbf{s}^{(k)^T}+\mathbf{s}^{(k)} \mathbf{y}^{(k)^T} H_k}{\mathbf{s}^{(k)^T} \mathbf{y}^{(k)}}$ 。置 $k=k+1$, 返回步骤 $(2) 。$
\end{algorithm*}
\newpage
\subsection*{非精确一维步长搜索}
\subsubsection*{非精确一维搜索}

找出满足某些适当条件的粗略近似解作为步长,提升算法的整体计算效率

\textbf{Wolfe-Powell conditions :}
$$
\begin{gathered}
\varphi(\alpha) \leq \varphi(0)+\rho \alpha \varphi^{\prime}(0) \\
\varphi^{\prime}(\alpha) \geq \sigma \varphi^{\prime}(0)
\end{gathered}
$$
其中$\rho \in (0, \frac{1}{2}), \sigma \in (\rho , 1)$是固定参数。

设$\hat{\alpha}_{k}$是使得$f(x^{(k)} + \alpha d^{(k)}) = f(x^{(k)})$的最小正数$\alpha$

\subsubsection*{基于Wolfe-Powell准则的非精确一维步长搜索}
\begin{enumerate}[(1)]
    \item 给定初始一维搜索区间 $[0, \bar{\alpha}]$, 以及 $\rho \in\left(0, \frac{1}{2}\right), \sigma \in(\rho, 1)$. 计算 $\varphi_0=\varphi(0)=f\left(x^{(k)}\right), \varphi_0^{\prime}=\varphi^{\prime}(0)=\nabla f\left(x^{(k)}\right)^T d^{(k)}$. 并令 $a_1=0, a_2=\bar{\alpha}, \varphi_1=\varphi_0, \varphi_1^{\prime}=\varphi_0^{\prime}$. 选取适当的 $\alpha \in\left(a_1, a_2\right)$.
    \item 计算 $\varphi=\varphi(\alpha)=f\left(x^{(k)}+\alpha d^{(k)}\right)$. 若 $\varphi(\alpha) \leq \varphi(0)+\rho \alpha \varphi^{\prime}(0)$, 则转到 第 (3) 步。否则,由 $\varphi_1, \varphi_1^{\prime}, \varphi$ 构造二次插值多项式 $p^{(1)}(t)$, 并得其极小点 $\hat{\alpha}$. 令 $a_2=\alpha, \alpha=\hat{\alpha}$, 重复第 (2) 步.
    \item 计算 $\varphi^{\prime}=\varphi^{\prime}(\alpha)=\nabla f\left(x^k+\alpha d^{(k)}\right)^T d^{(k)}$. 若 $\varphi^{\prime}(\alpha) \geq \sigma \varphi^{\prime}(0)$, 则输出 $\alpha_k=\alpha$, 并停止搜索. 否则由 $\varphi, \varphi^{\prime}, \varphi_1^{\prime}$ 构造两点二次插值多项式 $p^{(2)}(t)$ , 并求得极小点 $\hat{\alpha}$. 令 $a_1=\alpha, \alpha=\hat{\alpha}$, 返回第 $(2)$ 步.
\end{enumerate}

\section*{数据说明}
二维Rosenbrock函数
\section*{程序输入输出说明}
输入1、2选择DFP/BFGS,然后输入初始点。

输出值为迭代次数与最优解

\section*{程序测试结果}

\begin{table}[H]
    \begin{tabular}{llll}
    \textbf{DFP} &
      \textbf{epoch} &
      \textbf{x1, x2} &
      \textbf{f} \\ \hline
    \multicolumn{1}{|l|}{\textbf{0 0}} &
      \multicolumn{1}{l|}{\textbf{17}} &
      \multicolumn{1}{l|}{\textbf{0.9999999986994 0.9999999981660}} &
      \multicolumn{1}{l|}{\textbf{6.055373455850762e-17}} \\ \hline
    \multicolumn{1}{|l|}{\textbf{0.5 0.5}} &
      \multicolumn{1}{l|}{\textbf{16}} &
      \multicolumn{1}{l|}{\textbf{1.00000001203758 1.0000000247276}} &
      \multicolumn{1}{l|}{\textbf{1.8747613102236447e-16}} \\ \hline
    \multicolumn{1}{|l|}{\textbf{-1 1}} &
      \multicolumn{1}{l|}{\textbf{1}} &
      \multicolumn{1}{l|}{\textbf{1.0 1.0}} &
      \multicolumn{1}{l|}{\textbf{0.0}} \\ \hline
    \multicolumn{1}{|l|}{\textbf{2 -3}} &
      \multicolumn{1}{l|}{\textbf{37}} &
      \multicolumn{1}{l|}{\textbf{0.99999999503545 0.99999999346}} &
      \multicolumn{1}{l|}{\textbf{1.176336523561564e-15}} \\ \hline
    \multicolumn{1}{|l|}{\textbf{-100 150}} &
      \multicolumn{1}{l|}{} &
      \multicolumn{1}{l|}{\textbf{无法收敛}} &
      \multicolumn{1}{l|}{} \\ \hline
    \textbf{BFGS} &
      \textbf{epoch} &
      \textbf{x1, x2} &
      \textbf{f} \\ \hline
    \multicolumn{1}{|l|}{\textbf{0 0}} &
      \multicolumn{1}{l|}{\textbf{18}} &
      \multicolumn{1}{l|}{\textbf{0.99999999786 0.99999999574}} &
      \multicolumn{1}{l|}{\textbf{4.635405677357143e-18}} \\ \hline
    \multicolumn{1}{|l|}{\textbf{0.5 0.5}} &
      \multicolumn{1}{l|}{\textbf{15}} &
      \multicolumn{1}{l|}{\textbf{0.9999999995 0.9999999990}} &
      \multicolumn{1}{l|}{\textbf{2.4722121607796934e-19}} \\ \hline
    \multicolumn{1}{|l|}{\textbf{-1 1}} &
      \multicolumn{1}{l|}{\textbf{1}} &
      \multicolumn{1}{l|}{\textbf{1.0 1.0}} &
      \multicolumn{1}{l|}{\textbf{0.0}} \\ \hline
    \multicolumn{1}{|l|}{\textbf{2 -3}} &
      \multicolumn{1}{l|}{\textbf{26}} &
      \multicolumn{1}{l|}{\textbf{1.00000002922187 1.000000058506}} &
      \multicolumn{1}{l|}{\textbf{8.543076984615268e-16}} \\ \hline
    \multicolumn{1}{|l|}{\textbf{-100 150}} &
      \multicolumn{1}{l|}{\textbf{271}} &
      \multicolumn{1}{l|}{\textbf{0.99999999993 0.9999999999}} &
      \multicolumn{1}{l|}{\textbf{4.539167488746724e-19}} \\ \hline
    \end{tabular}
    \end{table}
\section*{分析总结}
不同的初始点也可能产生截然不同的结果,甚至无法收敛。

经过测试发现, BFGS法具有比DFP法更稳定的数值。
\end{document}